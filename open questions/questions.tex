\documentclass[a4paper,10pt]{article}
\usepackage[utf8]{inputenc}
\usepackage{listings}
\usepackage{color}
\usepackage{amsmath}
%opening
\title{}
\author{}

\definecolor{dkgreen}{rgb}{0,0.6,0}
\definecolor{gray}{rgb}{0.5,0.5,0.5}
\definecolor{mauve}{rgb}{0.58,0,0.82}

\lstset{frame=tb,
  language=Matlab,
  aboveskip=3mm,
  belowskip=3mm,
  showstringspaces=false,
  columns=flexible,
  basicstyle={\small\ttfamily},
  numbers=none,
  numberstyle=\tiny\color{gray},
  keywordstyle=\color{blue},
  commentstyle=\color{dkgreen},
  stringstyle=\color{mauve},
  breaklines=true,
  breakatwhitespace=true,
  tabsize=3
}

\begin{document}

\maketitle

\begin{itemize}
 \item The first and most fundamental question is, where the equation for the
simulated domar weights stems from
 \begin{lstlisting}
Out(N+1:2*N) = y' -
y'*diag(p)^epsilon*diag(A)^(epsilon-1)*diag(q)^(theta-
epsilon)*diag(1-alpha)* Omega *
diag(p)^(-theta)
- beta'*diag(p)^(-sigma)*C;
    \end{lstlisting}
stem from?

The formulation  \begin{align}
                 y_i &= p_i^{-\theta} \cdot \sum_{j=1}^{76} (\omega_{j,i} p_j^\varepsilon A_j^{\varepsilon -1} q_j^{\theta - \varepsilon} (1 - \alpha_j) y_j) - C p_i^{-\sigma} \beta_i
                \end{align}
(or similar), does not appear anywhere in the Paper.
\item Another thing that is unclear to me is how the formula for wages
                \begin{align}
                                          w_i &= p_i (A_i ^ {\frac{\varepsilon -1}{\epsilon}} \alpha_i ^ {\frac 1 \varepsilon} y_i ^ {\frac 1 \varepsilon} L_i ^ {\frac {-1}{\varepsilon}}) = p_i (A_i^{\varepsilon-1}\alpha_i y_i L_i^{-1})^{1/\varepsilon}
                \end{align}
      is derived.
\item The formula for prices \begin{lstlisting}
Out(1:N) =  p - (diag(A)^(epsilon-1)*(alpha.*w.^(1-epsilon)+(1-alpha).*q.^(1-epsilon))).^(1/(1-epsilon));
                             \end{lstlisting}
corrensponds to the CES-Production function in the paper on page 1185. However as far as my understanding goes the production function should output the quantity of items produced in sector $i$, not the prices of these items. Why can this be switched arround.
\item Normally CGE is compputed with some underlying optimization problem. Here however (and in the Covid-Paper) this simplifies to root finding of a nonlinear system of equations. The functions don't seem to be derivatives of any functions mentioned in the paper. So how exactly can this optimization problem be interchanged for a rootfinding problem?

\end{itemize}


\end{document}
